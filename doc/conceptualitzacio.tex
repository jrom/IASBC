%!TEX root = document.tex

En aquesta fase s'intenta adquirir la perspectiva de l'expert. De la interacció amb l'expert o experts en la recomanació de llibres n'hem d'extreure la dimensió final del domini i l'ontologia. A  la vegada, hem de descomposar el problema en subproblemes i realitzar un anàlisis per refinaments successius amb l'objectiu d'obtenir la relació jeràrquica entre les diferents fases de resolució fins als operadors de raonament més elementals. Cal, doncs, identificar quin és el flux de raonament necessari en la resolució del problema des del punt de vista de l'expert.

\section{Elements del domini}

Els primers elements a considerar són aquells que formen part de l'ontologia permanent del sistema, és a dir, els llibres, els autors i els gèneres.
\begin{itemize}
  \item Els llibres: El seu títol, el preu, l'any de publicació, el format, l'idioma, el número de pàgines, si és un best-seller o no ho és, l'enquadernació, l'edat a la qual està orientat, el número de de vendes, la trama, el vocabulari usat i si hi apareixen un gran nombre de personatges en són les seves característiques directes principals.
  \item Els autors: El seu nom, la nacionalitat i el seu any de naixement.
  \item Els gèneres: Dins del gènere de ficció, hi ha molts gèneres diferents. En guardem el seu nom.
\end{itemize}

També forma part del domini l'usuari: les seves característiques i preferències comunes i específiques (en funció de les respostes a les preguntes comunes).
\begin{itemize}
  \item Característiques: El seu nom, l'edat, el sexe, la llengua materna i el seu perfil d'ocupació.
  \item Preferències comunes: La freqüència i el temps de lectura, el gènere preferit dins de tots els disponibles al sistema, autor preferit dels que hi ha al sistema,  si li agrada experimentar amb nous gèneres, lloc habitual de lectura, tipus d'enquadernació preferida, importància del preu, preferència per llibres actuals, preferència per best-sellers, preferència per llibres gruixuts, preferència per llibres amb vocabulari senzill, preferència per llibres de trama enrevessada, preferència per llibres amb molts personatges, origen preferit de l'autor i llengua de lectura preferida.
  \item Preferències específiques: Preferència pels tipus de lletra grans, per aquells que no són ni nens petits ni persones grans; preferència per llibres lleugers per aquells que llegeixen fora de casa; preferència per llibres amb contingut explícit, per usuaris que no són nens; preferència per llibres més venuts per aquells que no són fans dels best sellers; preferència per una franja de preu determinada, per aquells a qui influeix el preu.
\end{itemize}

Per últim forma part també del domini les recomanacions emeses pel sistema, que estan ordenades per ordre de més apropiades a menys. Les tres millors són les que es retornen a l'usuari.

\section{Descomposició en subproblemes i flux de raonament}

Partim de la presumpció que tot llibre és recomanable per l'usuari en un inici. 

En la primera interacció amb l'usuari se li demanarà les característiques que el defineixen. Un cop assimilades, caldrà descartar, si s'escau, aquells llibres incompatibles amb el seu perfil. Per exemple, si es tracta d'una persona major de 60 anys es descarten aquells llibres de butxaca o amb lletra molt petita o bé si es tracta d'un infant se li recomanaran llibres de gènere infantil i per tant es descarten els altres. La resta de llibres no descartats podran veure modificada la seva puntuació de cara a la recomanació. Per exemple, si el perfil d'ocupació és acadèmic, la puntuació dels llibres clàssics o històrics es veurà incrementada.

Després es procedirà a fer les preguntes comunes. En funció de la rotunditat o incertesa de les respostes es procedirà sumar punts, restar-ne o bé esborrar aquells llibres en funció de les evidències i hipòtesis prèviament definides. Un cop respostes les preguntes comunes, es tindrà un \emph{ranking} limitat de llibres (sempre en funció del nivell de certesa de les respostes) i estarà ordenat de més recomanable a menys.

Finalment es passarà a realitzar les preguntes específiques, en funció del perfil de preferències que el sistema ha obtingut de l'usuari. Per exemple, l'usuari que ha respost que li influeix el preu se li pregunta entre quins marges es pot moure, o l'usuari que llegeix fora de casa se li pregunta fins a quin punt el pes del llibre pot resultar un problema. Igual que en l'anterior fase, les respostes que dóna l'usuari serviran per eliminar llibres o per variar-ne la seva puntuació. Quan s'han acabat de fer totes les preguntes s'analitzarà quins llibres són els més adequats i s'establirà l'ordre final.S'oferirà els 3 millors llibres a l'usuari com a recomanació.


\section{Abstraccions, Evidències i hipòtesis}

Després d'entrevistar de forma exhaustiva el nostre expert, n'hem extret unes quantes hipòtesis i evidències que s'acostumen a tenir en compte en un procés de recomanació de llibres i que serviran en la següent fase per elaborar el procés d'associació heurística i refinament. Aquí les detallarem per temes, en aquesta fase es mostren tal com l'expert ens les ha especificat.

\subsection{Edat de l'usuari}
Evidències:
\begin{itemize}
  \item Als nens se'ls recomana orientats a públic infantil.
  \item Als joves se'ls recomana orientats a públic juvenil.  
  \item Als adults se'ls recomana llibres orientats a públic adult.
  \item Als nens no se'ls recomanen llibres amb contingut explícit.
  \item Els llibres orientats a públic juvenil se'ls recomana als joves principalment i als adults en menys grau.
\end{itemize}

\subsection{Sexe}
Hipòotesis:
\begin{itemize}
  \item Als usuaris homes se'ls recomana en més grau la novel·la històrica i en menys grau la novel·la romàntica.
  \item Als usuaris dona se'ls recomana en més grau la novel·la thriller i la novel·la negra.
\end{itemize}

\subsection{Gèneres}
Evidències:
\begin{itemize}
  \item Se li recomanen preferentment els llibres del gèneres preferit de l'usuari.  
\end{itemize}
Hipòtesis:
\begin{itemize}
  \item Si a l'usuari li agrada experimentar, se li poden recomanar altres gèneres a part del seu preferit.  
  \item Als usuaris amb gènere preferit la ciència ficció també els pot agradar la fantasia (i viceversa).
  \item Als usuaris amb gènere preferit la novel·la històrica també els pot agradar els clàssics (i viceversa).
\end{itemize}

\subsection{Llocs de lectura}
Hipòtesis:
\begin{itemize}
  \item Els usuaris que llegeixen en transports públics prefereixen llibres de butxaca o tapa tova.
\end{itemize}

\subsection{Temps de lectura}
Hipòtesis:
\begin{itemize}
  \item Als usuaris que llegeixen molt se'ls recomanen llibres amb trama més complexa i vocabulari menys senzill. Altrament als usuaris que llegeixen poc.
  \item Als usuaris que dediquen més temps a llegir se'ls recomana llibres gruixuts.
\end{itemize}

\subsection{Influència del preu}
Evidències:
\begin{itemize}
  \item Els usuaris que els influeix molt el preu convé recomanar-lis llibres de butxaca i preguntar-lis en quina franja es poden moure.
\end{itemize}

\subsection{Format del llibre}
Hipòtesis:
\begin{itemize}
  \item Els usuaris que són persones grans prefereixen un tipus de lletra gran.
  \item Els usuaris que són nens prefereixen un tipus de lletra gran.
  \item Els usuaris que prefereixen un tipus de lletra gran no se'ls recomana llibres de butxaca.
  \item Els usuaris que prefereixen els llibres lleugers se'ls recomanen les edicions de butxaca.
  \item Els usuaris que prefereixen els llibres ben enquadernats se'ls recomana en més grau els llibres de tapa dura i els de tapa tova, i en menys grau els de butxaca.
  \item Hi ha usuaris que prefereixen llibres gruixuts, i cal preguntar-los-ho.
\end{itemize}

\subsection{Popularitat dels llibres}
Hipòtesis:
\begin{itemize}
  \item A tots els usuaris que no especifiquin el contrari, se'ls recomana un llibre més venut abans que un llibre menys venut.
  \item Els usuaris que no els agraden els best-sellers, no se'ls recomana llibres només perquè siguin els més venuts.
\end{itemize}

\subsection{Dates de publicació dels llibres}
Hipòtesis:
\begin{itemize}
  \item Usuari que li agraden les novetats se li recomanen els llibres de l'últim any.
  \item Usuari que li agraden els llibres actuals se'ls recomanen els llibres dels últims 10 anys.
\end{itemize}

\subsection{Estil del llibre}
Hipòtesis:
\begin{itemize}
  \item Als usuaris amb perfil acadèmic se'ls recomana novel·la històrica, clàssics o bé llibres amb trama complexa o vocabulari complex.
  \item Cal preguntar als usuaris si prefereixen llibres amb vocabulari senzill o més complicat i recomanar-lis llibres d'acord amb la seva resposta.
  \item Als usuaris que els agrada prendre notes al costat dels llibres se'ls recomana preferentment llibres de trama i vocabulari més complex.
\end{itemize}

\subsection{Autor preferit}
Hipòtesis:
\begin{itemize}
  \item L'usuari que té per autor preferit X, li agradaran els llibres del mateix estil que l'autor X i de semblant edat.
  \item Hi ha usuaris que prefereixen llibres escrits per autors d'una nacionalitat concreta, cal preguntar-ho i recomanar-lis d'acord amb les respostes.
\end{itemize}

\subsection{Llengües}
Hipòtesis:
\begin{itemize}
  \item Es recomanen amb més grau els llibres en la llengua materna de l'usuari.
  \item Cal preguntar als usuaris en quin idioma prefereixen llegir i recomanar-lis llibres d'aquell idioma.
\end{itemize}

\subsection{Vendes}
Abstraccions:
\begin{itemize}
  \item Els llibres amb menys de 500.000 exemplars venuts són relativament poc venuts.
  \item Els llibres entre 500.000 i 5.000.000 d'exemplars venuts són bastant venuts.
  \item Els llibres amb més de 5 milions d'exemplars venuts són molt venuts.
\end{itemize}


