%!TEX root = document.tex

En aquesta pràctica es planteja el problema de crear un Sistema Basat en el Coneixement per recomanar llibres adients als usuaris segons les seves preferències i característiques. En l'actualitat hi ha una enorme oferta de llibres de tots els tipus i gèneres, precisament per això resulta difícil per un usuari desinformat destriar entre tots ells quin és el llibre perfecte. Evidentment, el procés de recomanació d'un llibre és quelcom que va més enllà d'elegir gènere, tema, popularitat del llibre, etc. Per recomanar adequadament un llibre cal conèixer bé l'usuari i emular el procés de recomanació d'un llibreter que ens coneix de tota la vida.

En el nostre cas el domini de cerca es restringeix a les novel·les de ficció, tot i que dins d'aquest gènere hi ha molts subgèneres i llibres de molt diferent naturalesa. El Sistema Basat en el Coneixement que construïm recollirà eficientment informació de l'usuari i restringirà progressivament el domini. El sistema ha de comptar amb una base de dades d'instàncies de llibres significativa per poder donar respostes adequades.

En primer lloc, el sistema començarà identificant l'usuari, li realitzarà preguntes comunes i després preguntes específiques pel seu perfil, traurà conclusions esborrant opcions no adequades, realitzarà l'associació heurística segons els criteris extrets, es refinarà el resultat (categoritzant entre opcions parcialment recomanables, recomanables o molt recomanables) i, finalment, es mostrarà a l'usuari les opcions recomanades ordenades per adequació. En cas de tractar-se d'opcions parcialment recomanables s'indicarà quins criteris no compleixen.

\section{Metodologia per l'anàlisi del problema}

El problema que afrontem és un problema d'anàlisi i la metodologia que hem seguit es basa en un cicle de vida en cascada que també encaixa alhora amb la metodologia basada en el prototipat ràpid i el disseny incremental. Consta de les següents fases:

- Identificació del problema: On es determina la viabilitat de la construcció del SBC i la disponibilitat de les fonts de coneixement.
- Conceptualització: Descripció semiformal del coneixement del domini del problema i descomposició en subproblemes, segons la visió d'un expert.
- Formalització: En aquesta fase es pretén obtenir la visió del problema d'un enginyer del coneixement. Caldrà definir el mecanisme adequat de representació del coneixement.

Seguidament a aquestes fases, ve la implementació i les proves del sistema. Com que es tracta d'un procés incremental, s'ha començat amb un prototipat senzill, que s'ha anat extenent i refinant a mesura que s'ha anat provant, amb els canvis que això ha suposat al model final d'ontologia, la descomposició de subproblemes i el desplegament de casos.