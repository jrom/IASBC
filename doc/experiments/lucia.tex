%!TEX root = ../document.tex

\section{Experiment Lucia}

La Lucia és una noia italiana de 22 anys desocupada que ha estudiat Treball Social. És bastant llegidora, de fet llegeix dos hores al dia. Li interessa sobretot les trames socials, enrevessades als quals sovint emplena de notes o s'apunta informació en llibretes. Evita llibres cars i no li importa el format o l'enquadernació. Llegeix a tot arreu. No li importa que els llibres siguin best-sellers, ni el vocabulari que s'hi usi ni que abusin de contingut explícit. Tampoc li importa si són gruixuts o lleugers. Sempre li ha agradat Isabel Allende.

\begin{verbatim}
================================================================
Sistema basat en el coneixement recomanador de llibres

Nom? Lucia

Edat?  (número) 22

Sexe? (home dona) dona

Llengua materna? (catala castella altres) altres

Ocupació? (estudiant academic professional desocupat altres)
desocupat

En quina freqüència acostumes a llegir? (diaria setmanal mensual
indiferent) diaria

Quantes hores acostumes a llegir en aquesta freqüència? (número) 2

Gènere:  (classics fantasia ficcio historica negra romantica scifi
social terror thriller indiferent) social

T'agrada experimentar amb nous gèneres? (molt bastant indiferent
poc gens) bastant

On llegeixes habitualment? (casa exterior transport indiferent)
indiferent

Prefereixes els llibres ben enquadernats? (molt bastant indiferent
poc gens) poc

Prefereixes els llibres actuals? (molt bastant indiferent poc
gens) indiferent

T'agraden els best-sellers? (molt bastant indiferent poc gens)
indiferent

Estàs disposat a llegir llibres gruixuts? (molt bastant
indiferent poc gens) indiferent

Prefereixes que el vocabulari usat sigui senzill? (molt bastant
indiferent poc gens) indiferent

T'agrada prendre notes al costat dels llibres? (molt bastant
indiferent poc gens) molt

T'agraden els llibres amb molts personatges? (molt bastant
indiferent poc gens) indiferent

Quina és la teva preferència per la nacionalitat de l'autor?
(catala espanyol catala-espanyol estranger indiferent) indiferent

En quina llengua prefereixes llegir? (catala castella altres
indiferent) indiferent

Autor preferit: 
  1. Cap dels anteriors
  2. Stephenie Meyer
  3. Raymond Carver
  4. Paulo Coelho
  5. Gabriel García Márquez
  6. Thomas Pynchon
  7. Boris Pahor
  8. Julian Barnes
  9. Joseph Roth
  10. Jose Antonio Cebrián
  11. Parramon
  12. Juan Bonilla
  13. Quim Monzó
  14. Josep Pla
  15. Bernhard Schlink
  16. Khaled Hosseini
  17. Glen Cooper
  18. Mark Twain
  19. Roberto Bolaño
  20. Stieg Larsson
  21. Francisco Jimenez
  22. Xavier Montaño
  23. Robert Luis Stevenson
  24. Richard Yates
  25. Maria Àngels Anglada
  26. Xavier Bosch
  27. Camilla Lackberg
  28. Jane Bowles
  29. Thomas Pynchon
  30. Stefano Bordiglioni
  31. Donata Pizzato
  32. Antoine de Saint-Exupéry
  33. Laura Esquivel
  34. Sandra Cisneros
  35. Hector Abad Faciolince
  36. Italo Calvino
  37. Isaac Asimov
  38. Frank Herbert
  39. George Orwell
  40. J. R. R. Tolkien
  41. Clive Staples Lewis
  42. R. A. Salvatore
  43. Jules Verne
  44. Douglas Adams
  45. Orson Scott Card
  46. Arthur C. Clarke
  47. Vizconde de Lascano Tegui
  48. Carme Lafay
  49. J. K. Rowling
  50. Isabel Allende
  51. Dan Brown
  52. Carlos Ruiz Zafón
  53. Alice Sebold
Indica el numero corresponent: 50

Prefereixes els llibres lleugers? (molt bastant indiferent poc
gens) bastant

Prefereixes que la lletra sigui gran? (molt bastant indiferent
poc gens) bastant

Fins quan estàs disposat a gastar-te en un llibre? (número) 15

Prefereixes els llibres amb contingut explícit (violència o sexe)?
(molt bastant indiferent poc gens) indiferent
================================================================

Lucia, 
aquests llibres són la recomanació del SBC: 
================================================================
Titol:                   La casa de los espíritus
Autor:                   Isabel Allende
Preu:                    14.0 euros
ISBN:                    0060951303
Gènere:                  ficcio
Nacionalitat autor:      Chile
Any de publicació:       1995
Número de pàgines:       464
Enquadernació:           tapatova
Estil trama:             complexa
Estil vocabulari:        complexa
Quantiat de personatges: pocs
================================================================
Titol:                   I, Robot
Autor:                   Isaac Asimov
Preu:                    6.0 euros
ISBN:                    9755586932283
Gènere:                  scifi
Nacionalitat autor:      Rússia
Any de publicació:       1950
Número de pàgines:       211
Enquadernació:           butxaca
Estil trama:             complexa
Estil vocabulari:        complexa
Quantiat de personatges: pocs
================================================================
Titol:                   Dune
Autor:                   Frank Herbert
Preu:                    8.95 euros
ISBN:                    9711116932283
Gènere:                  scifi
Nacionalitat autor:      Estats Units
Any de publicació:       1965
Número de pàgines:       636
Enquadernació:           butxaca
Estil trama:             complexa
Estil vocabulari:        complexa
Quantiat de personatges: molts
================================================================
\end{verbatim}


Els llibres que li han tocat són diferents dels que està acostumada a llegir, però es mostra oberta receptiva i amb ganes de provar-los.
Li agrada especialment que siguin ben coneguts i que siguin barats i enrevessats. El primer llibre ja l'ha llegit i li va agradar molt.
