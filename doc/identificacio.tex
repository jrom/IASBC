%!TEX root = document.tex

\section{Identificació del problema}

La identificació del problema és la primera fase de l'anàlisi del problema. En primer lloc, consisteix en analitzar la viabilitat del sistema de coneixement (SBC) proposat. Cal justificar el sistema, demostrar que és adequat i que és capaç d'arribar a solucions satisfactòries. És també necessari especificar quines seran les fonts de coneixement que tindrem i quin d'aquest coneixement serà necessari pel sistema. Determinarem quins són els objectius del sistema, i quins tipus de sortides serà capaç d'oferir. També detallarem les hipòtesis prèvies que assumim.

\subsection{El SBC és viable}

Si es vol que el sistema sigui pràctic de cara a l'usuari, les bases de dades de llibres amb les que treballa han de ser molt grans, parlaríem de l'ordre de milions de llibres. Evidentment, si el sistema fos algorísmic, basat en qualsevol algorisme de cerca exhaustiva sobre tots aquests llibres, els resultats serien dolents i es trigaria molt en arribar a una solució. Com que el problema es defineix com un conjunt de restriccions i preferències, deduïm fàcilment que es pot resoldre amb un sistema basat en regles, com ara el que ofereix CLIPS.

En quant a la viabilitat econòmica del projecte, considerem que el mercat de les ventes de llibres per Internet es troba en constant creixement, on cada dia es presenta una nova plataforma de venta de llibres online. La majoria d'aquestes plataformes ofereixen avançats sistemes de cerca i molta informació a l'usuari, però no ofereixen sistemes intel·ligents de recomanació. Per tant, seria una aportació útil i innovativa pel mercat.

\subsection{Fonts de coneixement}

entrevista jordi romero + amazon.com + barnes and noble + la casa de libro + preguntar companys i amics

\subsection{Hipòtesis extretes}

\subsection{Objectius del sistema}


