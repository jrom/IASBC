%!TEX root = document.tex

\section{Procés d'implementació}

La implementació s'ha dut a terme en el següent ordre. S'ha començat creant una versió preliminar de l'ontologia mitjançant el programa Protégé. Un cop feta, s'han instanciat uns pocs llibres i autors d'exemple (també gràcies al programa Protégé). Llavors s'ha passat a implementar en CLIPS el mòdul del perfil d'usuari. Posteriorment s'ha passat a implementar el mòdul de preguntes comunes i finalment el de preguntes específiques. Mentre s'ha anat fent s'ha revisat constantment l'ontologia i les regles coincidint amb cada fase de la metodologia d'anàlisi del problema. Les regles s'han anat implementant progressivament i s'ha anat provant amb petits casos fixats cada vegada. Quan l'ontologia ha estat considerada definitiva s'ha creat un nombre més gran d'instàncies que podien representar tots els casos possibles del domini. A mesura que s'han seguit implementant noves regles, el procés d'associació heurística i refinament s'ha anat provant i assegurant-se que els resultats obtinguts eren coherents. Això s'emmarca dins de la metodologia de disseny incremental i prototipatge ràpid que hem decidit seguir des d'un inici.

\section{Ontologia}

\section{Repartiment dels mòduls}