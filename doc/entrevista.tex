%!TEX root = document.tex

Notes: Al final li he fet una entrevista de veritat a un amic (no és el meu pare). Ho completaré amb altres preguntes patillades i d'altres al meu pare.


\textbf{Què és el primer que preguntes quan algú et demana que li recomanis un llibre?}

  Quin tipus de llibre llegeixes normalment. Una pregunta molt bona, considero jo, és quins llibres t'has llegit més d'un cop.

\textbf{I si et diuen que no llegeixen gaire? Per on els faries començar?}

  Home, pots preguntar interessos en altres camps (com la televisió, cine, ...) per fer-te una idea.

\textbf{Recomanes coses diferents als homes i a les dones?}

  Depèn del cas, pero en general sí.

\textbf{Quins d'aquests punts creus que són més importants: l'edat del lector, el sexe, els gèneres que diu que són els seus preferits, l'estona que dedica a llegir, on acostuma a llegir (al llit, al metro, als bars), ...}

  Els gèneres preferits i el temps disponible (bàsicament perquè si hi dedica molta es fàcil que sigui millor lector i li pots recomanar coses mes difícils)

\textbf{Quines altres preguntes creus que ajuden a encertar-la recomanant llibres?}

Preguntaria si com a lector t'agrada que t'enganyin i et portin per terrenys desconeguts, o t'agraden les coses properes i conegudes.

\textbf{Quan algú et diu que és molt d'un gènere en concret, acostumes a intentar allunyar-lo d'aquest i suggerir-li coses diferents? O li fas cas i et cenyeixes al que ell et diu? (relacionat amb la pregunta anterior)}

 Jo li obriria el ventall. Sobretot si és un lector habitual i confio que acceptarà les alternatives.

 Una altra pregunta que crec que diferencia bastant als lectors és si prèn notes en el llibre o no.

\textbf{Creus que el tipus d'enquadernació afecta a l'elecció del llibre?}

  Jo crec que els lectors més reincidents poden ser una mica fetitxistes i preferir llibres de tapa dura, sobretot amb els seus autors preferits o llibres de capçalera. Una mica com passa amb els vinils.

\textbf{I com seria el teu llibre perfecte?}

  Llibre de poesia, tapa dura, un llibre gran però de poemes curts, complex, en català o castellà. la nacionalitat de l'autor no és massa important, però en cas de desconeguts tiro més per un país exòtic que proper. I si pot ser escrit al segle vint.
