%!TEX root = document.tex


\section{Secció}

% Fórmula $\Theta(P)$.
% Negreta \textbf{solIni2})
% Cursiva \emph{recorreguts en cercle}
% Referència~\ref{fig:{images/grafic_amb_control_final.png}}
% \imatgePetita{images/grafic_sense_control_final.png}{No es té en compte la distància entre la parada candidata i el final, les solucions poden caure en recorreguts en cercle.\texttt{} 


% \section{Intro}
% 
% \subsection{Blablabla}
% asdfasdf
% 
% \subsubsection{Lol}
% 
% Amb textmate es compila femt pometa-R
% 
% Ah, i has de deixar una línia en blanc entre dos paràgrafs si vols que surtin separats, sinó anirà a continuació...
% 
% Negreta: \textbf{text en negreta}
% 
% Cursiva: \textbf{Text en cursiva}

% Variable o similar: \codi{x.y}
% 
% \begin{lstlisting}[label=compi, caption=Comanda de compilació, language=Java]
% codi
% llarg
% \end{lstlisting}
% 
