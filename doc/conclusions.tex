%!TEX root = document.tex

La recomanació de llibres és un procés més complex i personal del que pot semblar aparentment. Per satisfer les necessitats de l'usuari no n'hi ha prou en fixar-se en les ses característiques disponibles dels llibres, sinó que caldria entrar en el seu contingut i analitzar-lo i crear una base de dades molt particular. En general, molts heurístics usats estan basats en estadístiques i poden no ser aplicables en tots els casos. Com més acurat es vol ser en el procés d'associació heurística més complex esdevé el problema.

Aquesta pràctica ens ha obert les portes d'un nou coneixement que, fins ara, gairebé no havíem tingut experiència a la carrera. La programació de llenguatges de regles, que s'havia vist lleugerament amb Prolog, obliga a canviar radicalment la perspectiva del programador. Cal codificar el programa pensant en les implicacions de les regles i no en procediments o funcions com estem acostumats. Gràcies a això, es pot expressar cada hipòtesi o limitació del Sistema Basat en el Coneixement de forma natural i senzilla. Tanmateix, cal dir que el llenguatge CLIPS resulta bastant ferragós i costa entrar-hi amb contacte. A més, creiem que és especialment difícil de debugar i costa sortir-se'n quan un es queda encallat en algun punt.



% by jordi
% definitivament és un llenguatge diferent als normals perquè permet programar pensant en immplicacións (->) i no en procediments o funcions
% la combinació d'un llenguatge OO amb aquesta particularitat permet expressar cada hipòtesi o limitació del SBC molt fàcilment (MENTIRA!)
% es podria fer el mateix en un petit DSL escrit en ruby
% és pesat de debugar  això si que ho pots dir  com a mancança
