\documentclass[a4paper]{report}
\usepackage{latexjrom}
\usepackage{fancyhdr}

\pagestyle{fancy}
\lhead{Pràctica de SBC}
\chead{} 
\rhead{\nouppercase{\leftmark}}
% \lfoot{\today}
% \cfoot{}
% \rfoot{\thepage}
\renewcommand{\headrulewidth}{0.4pt}
% \renewcommand{\footrulewidth}{0.4pt}

\renewcommand{\chaptermark}[1]{%
\markboth{\chaptername.  \thechapter. #1}{}}

\setlength{\parindent}{0pt}
\setlength{\parskip}{1ex plus 0.5ex minus 0.2ex}

\author{Bernat Farrero Badal \\Jordi Romero de Villalonga}
\title{Pràctica de Sistemes Basats en el Coneixement \\ Intel·ligència Artificial}
\date{12 d'Abril de 2010}

\begin{document}

\maketitle

\tableofcontents

\newpage

\chapter{Introducció}
%!TEX root = document.tex

En aquesta pràctica es planteja el problema de crear un Sistema Basat en el Coneixement per recomanar llibres adients als usuaris segons les seves preferències i característiques. En l'actualitat hi ha una enorme oferta de llibres de tots els tipus i gèneres, precisament per això resulta difícil per un usuari desinformat destriar entre tots ells quin és el llibre perfecte. Evidentment, el procés de recomanació d'un llibre és quelcom que va més enllà d'elegir gènere, tema, popularitat del llibre, etc. Per recomanar adequadament un llibre cal conèixer bé l'usuari i emular el procés de recomanació d'un llibreter que ens coneix de tota la vida.

En el nostre cas el domini de cerca es restringeix a les novel·les de ficció, tot i que dins d'aquest gènere hi ha molts subgèneres i llibres de molt diferent naturalesa. El Sistema Basat en el Coneixement que construïm recollirà eficientment informació de l'usuari i restringirà progressivament el domini. El sistema ha de comptar amb una base de dades d'instàncies de llibres significativa per poder donar respostes adequades.

En primer lloc, el sistema començarà identificant l'usuari, li realitzarà preguntes comunes i després preguntes específiques pel seu perfil, traurà conclusions esborrant opcions no adequades, realitzarà l'associació heurística segons els criteris extrets, es refinarà el resultat (categoritzant entre opcions parcialment recomanables, recomanables o molt recomanables) i, finalment, es mostrarà a l'usuari les opcions recomanades ordenades per adequació. En cas de tractar-se d'opcions parcialment recomanables s'indicarà quins criteris no compleixen.

\section{Metodologia per l'anàlisi del problema}

La metodologia que s'ha seguit està basada en un cicle de vida en cascada però també encaixa amb la metodologia basada en el prototipat ràpid i el disseny incremental. Aquesta consta de les següents fases:

- Identificació del problema: On es determina la viabilitat de la construcció del SBC i la disponibilitat de les fonts de coneixement.
- Conceptualització: Descripció semiformal del coneixement del domini del problema i descomposició en subproblemes, segons la visió d'un expert.
- Formalització: En aquesta fase es pretén obtenir la visió del problema d'un enginyer del coneixement. Caldrà definir el mecanisme adequat de representació del coneixement.

Seguidament a aquestes fases, ve la implementació i les proves del sistema. Com que es tracta d'un procés incremental, s'ha començat amb un prototipat senzill, que s'ha anat complicant i refinant a mesura que s'ha anat provant.

\chapter{Identificació del problema}
%!TEX root = document.tex

\section{Identificació del problema}

La identificació del problema és la primera fase de l'anàlisi del problema. En primer lloc, consisteix en analitzar la viabilitat del sistema de coneixement (SBC) proposat. Cal justificar el sistema, demostrar que és adequat i que és capaç d'arribar a solucions satisfactòries. És també necessari especificar quines seran les fonts de coneixement que tindrem i quin d'aquest coneixement serà necessari pel sistema. Determinarem quins són els objectius del sistema, i quins tipus de sortides serà capaç d'oferir. També detallarem les hipòtesis prèvies que assumim.

\subsection{El SBC és viable}

Si es vol que el sistema sigui pràctic de cara a l'usuari, les bases de dades de llibres amb les que treballa han de ser molt grans, parlaríem de l'ordre de milions de llibres. Evidentment, si el sistema fos algorísmic, basat en qualsevol algorisme de cerca exhaustiva sobre tots aquests llibres, els resultats serien dolents i es trigaria molt en arribar a una solució. Com que el problema es defineix com un conjunt de restriccions i preferències, deduïm fàcilment que es pot resoldre amb un sistema basat en regles, com ara el que ofereix CLIPS.

En quant a la viabilitat econòmica del projecte, considerem que el mercat de les ventes de llibres per Internet es troba en constant creixement, on cada dia es presenta una nova plataforma de venta de llibres en línia. La majoria d'aquestes plataformes ofereixen avançats sistemes de cerca i molta informació a l'usuari, però no ofereixen sistemes intel·ligents de recomanació. Per tant, seria una aportació útil i innovadora pel mercat.

\subsection{Fonts de coneixement}

La informació disponible dels llibres l'hem extret de diverses llibreries en línia. En particular, hem usat Amazon.com, Barnes&Noble i La Casa del Llibre. Les instàncies particulars les hem extret (mitjançant la programació d'un programa \emph{scrapper}) dels llocs web esmentats.

Evidentment, podem comptar només amb aquella informació que és pública: la descripció dels llibres i tots els seus camps característics i de l'autor, tot i que no podem conèixer a priori detalls personals de l'autor ni quines influències ha tingut per comparar-les amb l'usuari. Tampoc tenim la possibilitat que el sistema aprengui noves regles a força d'ús. Per exemple, seria interessant que el sistema pogués rebre una valoració positiva quan una recomanació és adequada o negativa quan no ho és, i corregir o reescriure automàticament algunes de les seves regles (sistema conegut com aprenentatge basat en explicacions).

Les hipòtesis i abstraccions heurístiques (definicionals, qualitatives i de generalització) sobre el problema les hem obtingut contactant amb un expert, que en aquest cas és un lector habitual. En particular, hem entrevistat a Jordi Romero (entrevista a l'annex) que publica un blog de lectura. També hem obtingut informació de la valoració que ens han fet companys i amics i de consells que hem trobat en llibreries en línia.

El coneixement necessari pel sistema es divideix en aquell coneixement.


\subsection{Evidències, hipòtesis i abstraccions heurístiques extretes}

\emph{En procés de discussió a /heuristics.md}

\textbf{Evidències: }

\textbf{Abstraccions heurístiques: }



\subsection{Objectius del sistema}

L'objectiu del sistema és generar una llista de \textbf{tres} llibres recomanats ordenat per grau d'adequació. Si algun d'aquests llibres és parcialment recomanat (perquè no compleix algun criteri) s'informarà l'usuari del criteri que no compleix. Però en principi i amb un domini suficientment gran, haurien d'existir almenys tres llibres recomanables per qualsevol combinació de preferències i restriccions de l'usuari.







\chapter{Conceptualització}
%!TEX root = document.tex

En aquesta fase s'intenta adquirir la perspectiva de l'expert. De la interacció amb l'expert o experts en la recomanació de llibres n'hem d'extreure la dimensió final del domini i l'ontologia. A  la vegada, hem de descomposar el problema en subproblemes i realitzar un anàlisis per refinaments successius amb l'objectiu d'obtenir la relació jeràrquica entre les diferents fases de resolució fins als operadors de raonament més elementals. Cal, doncs, identificar quin és el flux de raonament necessari en la resolució del problema des del punt de vista de l'expert.

\section{Elements del domini}

Els primers elements a considerar són aquells que formen part de l'ontologia permanent del sistema, és a dir, els llibres, els autors i els gèneres.
\begin{itemize}
  \item Els llibres: El seu títol, el preu, l'any d'edició, el número d'edicions, el format, l'idioma, el número de pàgines, la mesura de la lletra, si és un best-seller o no ho és, l'enquadernació, l'edat a la qual està orientat, el \emph{ranking} de ventes, la trama, el vocabulari usat i si hi apareixen un gran nombre de personatges en són les seves característiques directes principals.
  \item Els autors: El seu nom, l'estil en que escriuen, la nacionalitat i el seu gènere principal.
  \item Els gèneres: Dins del gènere de ficció, hi ha molts gèneres diferents. En guardem el seu nom.
\end{itemize}

També forma part del domini l'usuari: les seves característiques i preferències comunes i específiques (en funció de les respostes a les preguntes comunes).
\begin{itemize}
  \item Característiques: El seu nom, l'edat, el sexe, la llengua materna i el seu perfil d'ocupació.
  \item Preferències comunes: La freqüència i el temps de lectura, el gènere preferit d'una llista donada, autor preferit d'una llista donada,  si li agrada experimentar amb nous gèneres, lloc habitual de lectura, tipus d'enquadernació preferida, importància del preu, preferència per llibres actuals, preferència per best-sellers, preferència per llibres gruixuts, preferència per llibres amb vocabulari senzill, preferència per llibres de trama enrevessada, preferència per llibres amb molts personatges, origen preferit de l'autor i llengua de lectura preferida.
  \item Preferències específiques: Preferència pels tipus de lletra grans, per aquells que no són ni nens petits ni persones grans; preferència per llibres lleugers per aquells que llegeixen fora de casa; preferència per llibres amb contingut explícit, per usuaris que no són nens; preferència per llibres més venuts per aquells que no són fans dels best sellers; preferència per una franja de preu determinada, per aquells a qui influeix el preu.
\end{itemize}

Per últim forma part també del domini les recomanacions emeses pel sistema, que estan ordenades per ordre de més apropiades a menys. Les tres millors són les que es retornen a l'usuari.

\section{Descomposició en subproblemes i flux de raonament}

Partim de la presumpció que tot llibre és recomanable per l'usuari en un inici. 

En la primera interacció amb l'usuari se li demanarà les característiques que el defineixen. Un cop assimilades, caldrà descartar, si s'escau, aquells llibres incompatibles amb el seu perfil. Per exemple, si es tracta d'una persona major de 60 anys es descarten aquells llibres de butxaca o amb lletra molt petita o bé si es tracta d'un infant se li recomanaran llibres de gènere infantil i per tant es descarten els altres. La resta de llibres no descartats podran veure modificada la seva puntuació de cara a la recomanació. Per exemple, si el perfil d'ocupació és acadèmic, la puntuació dels llibres clàssics o històrics es veurà incrementada o pels desocupats s'entendrà que tenen més temps per llegir i s'incrementarà la puntuació dels llibres llargs. 

Després es procedirà a fer les preguntes comunes. En funció de la rotunditat o incertesa de les respostes es procedirà sumar punts, restar-ne o bé esborrar aquells llibres en funció de les evidències i hipòtesis prèviament definides. Un cop respostes les preguntes comunes, es tindrà un \emph{ranking} limitat de llibres (sempre en funció del nivell de certesa de les respostes) i estarà ordenat de més recomanable a menys.

Finalment es passarà a realitzar les preguntes específiques, en funció del perfil de preferències que el sistema ha obtingut de l'usuari. Per exemple, l'usuari que ha respost que li influeix el preu se li pregunta entre quins marges es pot moure, o l'usuari que llegeix fora de casa se li pregunta fins a quin punt el pes del llibre pot resultar un problema. Igual que en l'anterior fase, les respostes que dóna l'usuari serviran per eliminar llibres o per variar-ne la seva puntuació. Quan s'han acabat de fer totes les preguntes s'analitzarà quins llibres són els més adequats i s'establirà l'ordre final. Si hi ha 3 o més llibres a la llista s'oferirà els 3 millors llibres a l'usuari com a recomanació.


\section{Abstraccions, Evidències i hipòtesis}

Després d'entrevistar de forma exhaustiva el nostre expert, n'hem extret unes quantes hipòtesis i evidències que s'acostumen a tenir en compte en un procés de recomanació de llibres i que serviran en la següent fase per elaborar el procés d'associació heurística i refinament. Aquí les detallarem per temes, en aquesta fase es mostren tal com l'expert ens les ha especificat.

\subsection{Edat de l'usuari}
Evidències:
\begin{itemize}
  \item Als nens se'ls recomana orientats a públic infantil.
  \item Als joves se'ls recomana orientats a públic juvenil.  
  \item Als adults se'ls recomana llibres orientats a públic adult.
  \item Als nens no se'ls recomanen llibres amb contingut explícit.
  \item Els llibres orientats a públic juvenil se'ls recomana als joves principalment i als adults en menys grau.
\end{itemize}

\subsection{Sexe}
Hipòotesis:
\begin{itemize}
  \item Als usuaris homes se'ls recomana en més grau la novel·la històrica i en menys grau la novel·la romàntica.
  \item Als usuaris dona se'ls recomana en més grau la novel·la thriller i la novel·la negra.
\end{itemize}

\subsection{Gèneres}
Evidències:
\begin{itemize}
  \item Se li recomanen preferentment els llibres del gèneres preferit de l'usuari.  
\end{itemize}
Hipòtesis:
\begin{itemize}
  \item Si a l'usuari li agrada experimentar, se li poden recomanar altres gèneres a part del seu preferit.  
  \item Als usuaris amb gènere preferit la ciència ficció també els pot agradar la fantasia (i viceversa).
  \item Als usuaris amb gènere preferit la novel·la històrica també els pot agradar els clàssics (i viceversa).
\end{itemize}

\subsection{Llocs de lectura}
Hipòtesis:
\begin{itemize}
  \item Els usuaris que llegeixen en transports públics prefereixen llibres de butxaca o tapa tova.
\end{itemize}

\subsection{Temps de lectura}
Hipòtesis:
\begin{itemize}
  \item Als usuaris que llegeixen molt se'ls recomanen llibres amb trama més complexa i vocabulari menys senzill. Altrament als usuaris que llegeixen poc.
  \item Als usuaris que dediquen més temps a llegir se'ls recomana llibres gruixuts.
\end{itemize}

\subsection{Influència del preu}
Evidències:
\begin{itemize}
  \item Els usuaris que els influeix molt el preu convé recomanar-lis llibres de butxaca i preguntar-lis en quina franja es poden moure.
\end{itemize}

\subsection{Format del llibre}
Hipòtesis:
\begin{itemize}
  \item Els usuaris que són persones grans prefereixen un tipus de lletra gran.
  \item Els usuaris que són nens prefereixen un tipus de lletra gran.
  \item Els usuaris que prefereixen un tipus de lletra gran no se'ls recomana llibres de butxaca.
  \item Els usuaris que prefereixen els llibres lleugers se'ls recomanen les edicions de butxaca.
  \item Els usuaris que prefereixen els llibres ben enquadernats se'ls recomana en més grau els llibres de tapa dura i els de tapa tova, i en menys grau els de butxaca.
  \item Hi ha usuaris que prefereixen llibres gruixuts, i cal preguntar-los-ho.
\end{itemize}

\subsection{Popularitat dels llibres}
Hipòtesis:
\begin{itemize}
  \item A tots els usuaris que no especifiquin el contrari, se'ls recomana un llibre més venut abans que un llibre menys venut.
  \item Els usuaris que no els agraden els best-sellers, no se'ls recomana llibres només perquè siguin els més venuts.
\end{itemize}

\subsection{Dates de publicació dels llibres}
Hipòtesis:
\begin{itemize}
  \item Usuari que li agraden les novetats se li recomanen els llibres de l'últim any.
  \item Usuari que li agraden els llibres actuals se'ls recomanen els llibres dels últims 10 anys.
\end{itemize}

\subsection{Estil del llibre}
Hipòtesis:
\begin{itemize}
  \item Als usuaris amb perfil acadèmic se'ls recomana novel·la històrica, clàssics o bé llibres amb trama complexa o vocabulari complex.
  \item Cal preguntar als usuaris si prefereixen llibres amb vocabulari senzill o més complicat i recomanar-lis llibres d'acord amb la seva resposta.
  \item Als usuaris que els agrada prendre notes al costat dels llibres se'ls recomana preferentment llibres de trama i vocabulari més complex.
\end{itemize}

\subsection{Autor preferit}
Hipòtesis:
\begin{itemize}
  \item L'usuari que té per autor preferit X, li agradaran els llibres del mateix estil que l'autor X i de semblant edat.
  \item Hi ha usuaris que prefereixen llibres escrits per autors d'una nacionalitat concreta, cal preguntar-ho i recomanar-lis d'acord amb les respostes.
\end{itemize}

\subsection{Llengües}
Hipòtesis:
\begin{itemize}
  \item Es recomanen amb més grau els llibres en la llengua materna de l'usuari.
  \item Cal preguntar als usuaris en quin idioma prefereixen llegir i recomanar-lis llibres d'aquell idioma.
\end{itemize}

\subsection{Vendes}
Abstraccions:
\begin{itemize}
  \item Els llibres amb menys de 500.000 exemplars venuts són relativament poc venuts.
  \item Els llibres entre 500.000 i 5.000.000 d'exemplars venuts són bastant venuts.
  \item Els llibres amb més de 5 milions d'exemplars venuts són molt venuts.
\end{itemize}




\chapter{Formalització del problema}
%!TEX root = document.tex

\section{Formalisme de representació del coneixement}

Per representar la informació del banc de llibres usarem una ontologia. En la fase anterior hem estudiat quins són els elements del nostre domini i ara determinem quins d'aquests elements seran representats per l'ontologia. En el cas del nostre problema en concret, l'ontologia la formaran els llibres, els gèneres i els autors. Tota aquella informació permanent que l'adquirent del nostre SBC mantindrà, usarà i actualitzarà estarà estructurada dins l'ontologia. Tanmateix, aquella informació provisional com és el perfil de l'usuari i les recomanacions emeses pel sistema es trobaran en el propi llenguatge de regles. El detall concret de l'ontologia es troba en el capítol \textbf{XXX} ontologia.

Per programar l'aplicació s'usarà el programa basat en regles CLIPS. CLIPS és una eina per la producció i execució de sistemes experts escrita en C. Les regles descrites s'expliquen en l'apartat \textbf{XXX} d'aquest document.

El problema proposat és un problema d'anàlisis. Parteix d'una situació inicial on tot és solució i mitjançant l'abstracció de dades, l'aplicació de regles (associació heurística) i posterior refinament s'arriba a una solució bona per l'usuari. Aquesta metodologia de resolució de problemes s'anomena classificació heurística i en els propers apartats s'explicarà amb més detalls.

\section{Mida de l'espai de cerca }

L'espai de cerca consta bàsicament dels llibres del sistema. Com que els llibres s'hauran de recórrer per cada criteri que es pregunta, l'espai de cerca serà equivalent a la multiplicació dels criteris pel nombre de llibres. Evidentment, el fet d'eliminar llibres de la llista reduirà l'espai de cerca. Per tant, un usuari que sigui molt taxatiu descartant llibres farà minvar ràpidament el domini de llibres a recomanar.

\section{Metodologia de resolució del problema}

La metodologia de resolució del problema que s'usa és la \textbf{classificació heurística}. La seva característica principal consisteix en el fet que l'expert escull una categoria concreta d'un conjunt de solucions prèviament enumerat. Aquest és precisament el cas que ens afecta. Es parteix d'un conjunt de solucions que són tots els llibres i mitjançant un procés d'anàlisi i de classificació (basat amb les respostes de l'usuari) anem descartat i avaluant el domini.

La classificació heurística es divideix en tres etapes. La primera etapa és l'\textbf{abstracció de les dades}. Aquesta pot ser definicional, per exemple un usuari que li agrada prendre notes al costat d'un llibre es defineix com a usuari complex, o bé pot ser qualitativa, per exemple l'edat de l'usuari al sistema es classifica com a nen, jove, adult o persona gran. Per últim pot ser de generalització, com és el cas dels usuaris a qui agraden els best-sellers, s'enregistren com a amants dels top-ventes. La segona etapa a realitzar és l'\textbf{associació heurística} i es realitzarà mitjançant les regles definides de les hipòtesis i evidències extretes de la interacció amb l'expert en l'apartat de conceptualització. Per últim, resta l'etapa de \textbf{refinament}, on s'acabaran ordenant i escollint els millors llibres per retornar a l'usuari.

Esquema del procés i els blocs de raonament:

\textbf{**Esquema**}

\section{Tractament de la incertesa}

Per fer un sistema que sigui natural per l'usuari s'ha d'acceptar la incertesa habitual del món real. Això vol dir que les respostes a les preguntes no han de ser si o no, sinó que s'ha d'oferir un més ampli espectre possible de certesa. En el nostre cas, aquesta incertesa l'introduïm en el segon i tercer blocs de preguntes. En el primer no s'accepta perquè preguntem valors deterministes, com per exemple el nom i l'edat. El segon i tercer bloc de preguntes, en canvi, s'admetran les respostes Molt, Bastant, Indiferent, Poc o Gens. En cas que sigui gens generalment es procedirà a eliminar els llibres de la llista, si es tracta de poc es restarà punts, si el criteri és indiferent es deixarà igual, i si és bastant o molt es sumaran punts als llibres coincidents.

Per tal de no cometre errors en el refinament, el que hem fet en una segona iteració del procés incremental de desenvolupament del nostre sistema és mantenir dos atributs de valoració dels llibres. Un per quantificar les valoracions negatives i l'altre per quantificar les valoracions positives. Els casos en que la resposta és Indiferent no aporta cap informació nova al sistema.




\chapter{Implementació}
%!TEX root = document.tex

Aquí va explicada la implementació

\chapter{Ontologia}
\label{ontologia}
\section{Fases de desenvolupament de l'ontologia}
(pàg 93 llibre)


\chapter{Descripció dels mòduls}
%!TEX root = document.tex
\label{moduls}

La implementació del sistema basat en el coneixement s'ha dividit en mòduls per tal que cada tasca estigui en una secció de codi i el programa salti d'un mòdul a l'altre depenent de la tasca a realitzar.

A continuació es llisten els mòduls implementats junt amb una breu descripció de la seva tasca.

\section{Principal}

És el mòdul que s'executa primer. Es defineixen les funcions auxiliars, templates, assumpcions inicials i es mostra un missatge de benvinguda.

\section{Perfil Lector}

En el primer mòdul on es fan preguntes, se li demana a l'usuari que respongui un seguit de preguntes per formar un perfil de lector. Se li demana el nom per cortesia, edat, sexe, llengua materna i ocupació. Amb aquestes dades després podrem decidir fer determinades preguntes o prendre determinades conclusions, com expliquem a l'apartat d'Evidències i Hipòtesis.


\section{Preguntes Comunes}

En aquest mòdul es demana a l'usuari que respongui un segut de preguntes per permetre al sistema obtenir informació sobre preferències en els habits de lectura, gèneres preferits i conceptes relacionats amb el domini del problema.

\section{Preguntes Específiques}

Depenent del perfil del lector, o de respostes en el mòdul anterior, es procedeix a concretar preferències o fer preguntes que s'obvien si el perfil és un altre.

\section{Associacions Incondicionals}

Lligat amb l'apartat anterior, aquí s'extreuen conclusions a partir d'informació del perfil o de preguntes comunes, com per exemple quan no cal fer una pregunta específica perquè una característica del perfil del lector ja implica una conclusió.

\section{Esborrar instàncies}

Quan no es compleixen els requisits mínims establerts per l'usuari s'esborra la instància del llibre. D'aquesta manera es redueix l'espai de cerca per al pas posterior.

\section{Associacions heurístiques}

En el mòdul on s'apliquen les associacions heurístiques és on es creen totes les possibles recomanacions de llibres i les respostes de l'usuari classifiquen aquestes recomanacions. Es qualifiquen diferents indicadors per finalment en el proper mòdul obtenir-ne l'ordre final.

\section{Refinament}

Després d'haver aplicat les associacions heurístiques sobre el conjunt inicial de recomanacions, en aquest mòdul s'ordenen valorant els diferents indicadors de cada recomanació i se n'escullen les tres millors, o les millors en cas de que s'hagin descartat gairebé tots els llibres. En aquest mòdul s'ensenya per pantalla a l'usuari la solució del problema.


\chapter{Casos de prova}
%!TEX root = document.tex

Introducció
==========

================================================================
################################################################
Bernat #########################################################
################################################################
================================================================


Perfil
======

El Bernat és un estudiant català d'informàtica amant de les novel·les de ciència ficció (tot i que li agrada experimentar amb nous gèneres). Té poc temps per llegir, així que llegeix unes 4 hores al mes generalment, excepte en períodes de vacances. Llegeix a tot arreu, al bus o a casa tot i que no li preocupa especialment el pes del llibre. No el motiven ni desmotiven especialment els llibres més venuts, i intenta evitar els \emph{totxos} degut a la manca de temps. Li agrada prendre notes d'alguna que altra frase interessant que li ha quedat i no li importa la llengua en que estigui escrit el llibre ni l'origen de l'autor. El seu autor preferit ha sigut sempre Tolkien i intenta evitar els llibres que costin més de 15€. 


Sistema basat en el coneixement recomanador de llibres

Nom? Bernat

Edat?  (número) 24

Sexe? (home dona) home

Llengua materna? (catala castella altres) catala

Ocupació? (estudiant academic professional desocupat altres) estudiant

En quina freqüència acostumes a llegir? (diaria setmanal mensual indiferent) mensual

Quantes hores acostumes a llegir en aquesta freqüència? (número) 4

Gènere:  (classics fantasia ficcio historica negra romantica scifi social terror thriller indiferent) scifi

T'agrada experimentar amb nous gèneres? (molt bastant indiferent poc gens) bastant

On llegeixes habitualment? (casa exterior transport indiferent) indiferent

Prefereixes els llibres ben enquadernats? (molt bastant indiferent poc gens) poc

Prefereixes els llibres actuals? (molt bastant indiferent poc gens) bastant   

T'agraden els best-sellers? (molt bastant indiferent poc gens) indiferent

Estàs disposat a llegir llibres gruixuts? (molt bastant indiferent poc gens) poc

Prefereixes que el vocabulari usat sigui senzill? (molt bastant indiferent poc gens) poc

T'agrada prendre notes al costat dels llibres? (molt bastant indiferent poc gens) bastant

T'agraden els llibres amb molts personatges? (molt bastant indiferent poc gens) indiferent

Quina és la teva preferència per la nacionalitat de l'autor? (catala espanyol catala-espanyol estranger indiferent) indiferent

En quina llengua prefereixes llegir? (catala castella altres indiferent) indiferent

Autor preferit: 
  1. Cap dels anteriors
  2. Stephenie Meyer
  3. Raymond Carver
  4. Paulo Coelho
  5. Gabriel García Márquez
  6. Thomas Pynchon
  7. Boris Pahor
  8. Julian Barnes
  9. Joseph Roth
  10. Jose Antonio Cebrián
  11. Parramon
  12. Juan Bonilla
  13. Quim Monzó
  14. Josep Pla
  15. Bernhard Schlink
  16. Khaled Hosseini
  17. Glen Cooper
  18. Mark Twain
  19. Roberto Bolaño
  20. Stieg Larsson
  21. Francisco Jimenez
  22. Xavier Montaño
  23. Robert Luis Stevenson
  24. Richard Yates
  25. Maria Àngels Anglada
  26. Xavier Bosch
  27. Camilla Lackberg
  28. Jane Bowles
  29. Thomas Pynchon
  30. Stefano Bordiglioni
  31. Donata Pizzato
  32. Antoine de Saint-Exupéry
  33. Laura Esquivel
  34. Sandra Cisneros
  35. Hector Abad Faciolince
  36. Italo Calvino
  37. Isaac Asimov
  38. Frank Herbert
  39. George Orwell
  40. J. R. R. Tolkien
  41. Clive Staples Lewis
  42. R. A. Salvatore
  43. Jules Verne
  44. Douglas Adams
  45. Orson Scott Card
  46. Arthur C. Clarke
  47. Vizconde de Lascano Tegui
  48. Carme Lafay
  49. J. K. Rowling
  50. Isabel Allende
  51. Dan Brown
  52. Carlos Ruiz Zafón
  53. Alice Sebold
Indica el numero corresponent: 40

Prefereixes els llibres lleugers? (molt bastant indiferent poc gens) indiferent

Prefereixes que la lletra sigui gran? (molt bastant indiferent poc gens) indiferent

Fins quan estàs disposat a gastar-te en un llibre? (número) 15

Prefereixes els llibres amb contingut explícit (violència o sexe)? (molt bastant indiferent poc gens) bastant
================================================================

Bernat, 
aquests llibres són la recomanació del SBC: 
================================================================
Titol:                   Tots tenim secrets
Autor:                   Carme Lafay
Preu:                    11.8 euros
ISBN:                    8497916824
Gènere:                  scifi
Nacionalitat autor:      Espanya
Any de publicació:       2007
Número de pàgines:       230
Enquadernació:           tapatova
Estil trama:             simple
Estil vocabulari:        simple
Quantiat de personatges: molts
================================================================
Titol:                   El restaurant de la fi del món
Autor:                   Douglas Adams
Preu:                    6.0 euros
ISBN:                    9788433912411
Gènere:                  scifi
Nacionalitat autor:      Regne Unit
Any de publicació:       2006
Número de pàgines:       208
Enquadernació:           butxaca
Estil trama:             simple
Estil vocabulari:        simple
Quantiat de personatges: molts
================================================================
Titol:                   El Hobbit
Autor:                   J. R. R. Tolkien
Preu:                    8.0 euros
ISBN:                    9712346932283
Gènere:                  fantasia
Nacionalitat autor:      Regne Unit
Any de publicació:       1937
Número de pàgines:       220
Enquadernació:           butxaca
Estil trama:             simple
Estil vocabulari:        complexa
Quantiat de personatges: molts

================================================================


Conclusió:

Els llibres s'ajusten molt al seu perfil, i de fet 2/3 són llibres que ja ha llegit. Són llibres de preu ajustat, del gènere de ciència ficció o fantasia, enrevessats i amb molts personatges, tal i com li agrada. En el seu cas creiem que la recomanació és existosa.



================================================================
################################################################
Lucia ##########################################################
################################################################
================================================================


Perfil
=======

La Lucia és una noia italiana de 22 anys desocupada que ha estudiat Treball Social. És bastant llegidora, de fet llegeix dos hores al dia. Li interessa sobretot les trames socials, enrevessades als quals sovint emplena de notes o s'apunta informació en llibretes. Evita llibres cars i no li importa el format o l'enquadernació. Llegeix a tot arreu. No li importa que els llibres siguin best-sellers, ni el vocabulari que s'hi usi ni que abusin de contingut explícit. Tampoc li importa si són gruixuts o lleugers. Sempre li ha agradat Isabel Allende.


Sistema basat en el coneixement recomanador de llibres

Nom? Lucia

Edat?  (número) 22

Sexe? (home dona) dona

Llengua materna? (catala castella altres) altres

Ocupació? (estudiant academic professional desocupat altres) desocupat

En quina freqüència acostumes a llegir? (diaria setmanal mensual indiferent) diaria

Quantes hores acostumes a llegir en aquesta freqüència? (número) 2

Gènere:  (classics fantasia ficcio historica negra romantica scifi social terror thriller indiferent) social

T'agrada experimentar amb nous gèneres? (molt bastant indiferent poc gens) bastant

On llegeixes habitualment? (casa exterior transport indiferent) indiferent

Prefereixes els llibres ben enquadernats? (molt bastant indiferent poc gens) poc

Prefereixes els llibres actuals? (molt bastant indiferent poc gens) indiferent

T'agraden els best-sellers? (molt bastant indiferent poc gens) indiferent

Estàs disposat a llegir llibres gruixuts? (molt bastant indiferent poc gens) indiferent

Prefereixes que el vocabulari usat sigui senzill? (molt bastant indiferent poc gens) indiferent

T'agrada prendre notes al costat dels llibres? (molt bastant indiferent poc gens) molt

T'agraden els llibres amb molts personatges? (molt bastant indiferent poc gens) indiferent

Quina és la teva preferència per la nacionalitat de l'autor? (catala espanyol catala-espanyol estranger indiferent) indiferent

En quina llengua prefereixes llegir? (catala castella altres indiferent) indiferent

Autor preferit: 
  1. Cap dels anteriors
  2. Stephenie Meyer
  3. Raymond Carver
  4. Paulo Coelho
  5. Gabriel García Márquez
  6. Thomas Pynchon
  7. Boris Pahor
  8. Julian Barnes
  9. Joseph Roth
  10. Jose Antonio Cebrián
  11. Parramon
  12. Juan Bonilla
  13. Quim Monzó
  14. Josep Pla
  15. Bernhard Schlink
  16. Khaled Hosseini
  17. Glen Cooper
  18. Mark Twain
  19. Roberto Bolaño
  20. Stieg Larsson
  21. Francisco Jimenez
  22. Xavier Montaño
  23. Robert Luis Stevenson
  24. Richard Yates
  25. Maria Àngels Anglada
  26. Xavier Bosch
  27. Camilla Lackberg
  28. Jane Bowles
  29. Thomas Pynchon
  30. Stefano Bordiglioni
  31. Donata Pizzato
  32. Antoine de Saint-Exupéry
  33. Laura Esquivel
  34. Sandra Cisneros
  35. Hector Abad Faciolince
  36. Italo Calvino
  37. Isaac Asimov
  38. Frank Herbert
  39. George Orwell
  40. J. R. R. Tolkien
  41. Clive Staples Lewis
  42. R. A. Salvatore
  43. Jules Verne
  44. Douglas Adams
  45. Orson Scott Card
  46. Arthur C. Clarke
  47. Vizconde de Lascano Tegui
  48. Carme Lafay
  49. J. K. Rowling
  50. Isabel Allende
  51. Dan Brown
  52. Carlos Ruiz Zafón
  53. Alice Sebold
Indica el numero corresponent: 50

Prefereixes els llibres lleugers? (molt bastant indiferent poc gens) bastant

Prefereixes que la lletra sigui gran? (molt bastant indiferent poc gens) bastant

Fins quan estàs disposat a gastar-te en un llibre? (número) 15

Prefereixes els llibres amb contingut explícit (violència o sexe)? (molt bastant indiferent poc gens) indiferent
================================================================

Lucia, 
aquests llibres són la recomanació del SBC: 
================================================================
Titol:                   La casa de los espíritus
Autor:                   Isabel Allende
Preu:                    14.0 euros
ISBN:                    0060951303
Gènere:                  ficcio
Nacionalitat autor:      Chile
Any de publicació:       1995
Número de pàgines:       464
Enquadernació:           tapatova
Estil trama:             complexa
Estil vocabulari:        complexa
Quantiat de personatges: pocs
================================================================
Titol:                   I, Robot
Autor:                   Isaac Asimov
Preu:                    6.0 euros
ISBN:                    9755586932283
Gènere:                  scifi
Nacionalitat autor:      Rússia
Any de publicació:       1950
Número de pàgines:       211
Enquadernació:           butxaca
Estil trama:             complexa
Estil vocabulari:        complexa
Quantiat de personatges: pocs
================================================================
Titol:                   Dune
Autor:                   Frank Herbert
Preu:                    8.95 euros
ISBN:                    9711116932283
Gènere:                  scifi
Nacionalitat autor:      Estats Units
Any de publicació:       1965
Número de pàgines:       636
Enquadernació:           butxaca
Estil trama:             complexa
Estil vocabulari:        complexa
Quantiat de personatges: molts

Conclusió
==========

Els llibres que li han tocat són diferents dels que està acostumada a llegir, però es mostra oberta receptiva i amb ganes de provar-los.
Li agrada especialment que siguin ben coneguts i que siguin barats i enrevessats. El primer llibre ja l'ha llegit i li va agradar molt.


================================================================
################################################################
Gerard  ########################################################
################################################################
================================================================

Perfil
======

El Gerard és un noi de 15 anys que just ha descobert el món de la lectura. Li agrada llegir totxos com més grans millors, i sobretot li encanten les sèries de novel·les famoses, com \emph{El Codi Davinci} o similars. Llegeix bastant i diu que de tant en tant es pot permetre comprar-se llibres de 20€, tot i que sovint li regalen o li deixa sa mare. Li agraden els llibres "macos", ben enquadernats i amb lletra gran. Confia bastant amb el número de vendes si té algun dubte sobre un llibre i, si pot, escull autors estrangers ja que creu que "acostumen a ser millors que el d'aquí!".


Sistema basat en el coneixement recomanador de llibres

Nom? Gerard

Edat?  (número) 15

Sexe? (home dona) home

Llengua materna? (catala castella altres) catala

Ocupació? (estudiant academic professional desocupat altres) estudiant

En quina freqüència acostumes a llegir? (diaria setmanal mensual indiferent) setmanal

Quantes hores acostumes a llegir en aquesta freqüència? (número) 2

Gènere:  (classics fantasia ficcio historica negra romantica scifi social terror thriller indiferent) historica

T'agrada experimentar amb nous gèneres? (molt bastant indiferent poc gens) indiferent

On llegeixes habitualment? (casa exterior transport indiferent) casa

Prefereixes els llibres ben enquadernats? (molt bastant indiferent poc gens) bastant

Prefereixes els llibres actuals? (molt bastant indiferent poc gens) bastant

T'agraden els best-sellers? (molt bastant indiferent poc gens) molt

Estàs disposat a llegir llibres gruixuts? (molt bastant indiferent poc gens) molt

Prefereixes que el vocabulari usat sigui senzill? (molt bastant indiferent poc gens) indiferent

T'agrada prendre notes al costat dels llibres? (molt bastant indiferent poc gens) gens

T'agraden els llibres amb molts personatges? (molt bastant indiferent poc gens) poc

Quina és la teva preferència per la nacionalitat de l'autor? (catala espanyol catala-espanyol estranger indiferent) estranger

En quina llengua prefereixes llegir? (catala castella altres indiferent) catala

Autor preferit: 
  1. Cap dels anteriors
  2. Stephenie Meyer
  3. Raymond Carver
  4. Paulo Coelho
  5. Gabriel García Márquez
  6. Thomas Pynchon
  7. Boris Pahor
  8. Julian Barnes
  9. Joseph Roth
  10. Jose Antonio Cebrián
  11. Parramon
  12. Juan Bonilla
  13. Quim Monzó
  14. Josep Pla
  15. Bernhard Schlink
  16. Khaled Hosseini
  17. Glen Cooper
  18. Mark Twain
  19. Roberto Bolaño
  20. Stieg Larsson
  21. Francisco Jimenez
  22. Xavier Montaño
  23. Robert Luis Stevenson
  24. Richard Yates
  25. Maria Àngels Anglada
  26. Xavier Bosch
  27. Camilla Lackberg
  28. Jane Bowles
  29. Thomas Pynchon
  30. Stefano Bordiglioni
  31. Donata Pizzato
  32. Antoine de Saint-Exupéry
  33. Laura Esquivel
  34. Sandra Cisneros
  35. Hector Abad Faciolince
  36. Italo Calvino
  37. Isaac Asimov
  38. Frank Herbert
  39. George Orwell
  40. J. R. R. Tolkien
  41. Clive Staples Lewis
  42. R. A. Salvatore
  43. Jules Verne
  44. Douglas Adams
  45. Orson Scott Card
  46. Arthur C. Clarke
  47. Vizconde de Lascano Tegui
  48. Carme Lafay
  49. J. K. Rowling
  50. Isabel Allende
  51. Dan Brown
  52. Carlos Ruiz Zafón
  53. Alice Sebold
Indica el numero corresponent: 51

Si estàs entre dos llibres, prefereixes aquell que es ven més? (molt bastant indiferent poc gens) bastant

Prefereixes que la lletra sigui gran? (molt bastant indiferent poc gens) bastant

Fins quan estàs disposat a gastar-te en un llibre? (número) 20

Prefereixes els llibres amb contingut explícit (violència o sexe)? (molt bastant indiferent poc gens) indiferent
================================================================

Gerard, 
aquests llibres són la recomanació del SBC: 
================================================================
Titol:                   Eclipse
Autor:                   Stephenie Meyer
Preu:                    16.0 euros
ISBN:                    1603960228
Gènere:                  terror
Nacionalitat autor:      Estats Units
Any de publicació:       2007
Número de pàgines:       637
Enquadernació:           tapatova
Bestseller
Estil trama:             complexa
Estil vocabulari:        complexa
Quantiat de personatges: pocs
================================================================
Titol:                   Luna Nueva
Autor:                   Stephenie Meyer
Preu:                    15.0 euros
ISBN:                    9705800235
Gènere:                  terror
Nacionalitat autor:      Estats Units
Any de publicació:       2007
Número de pàgines:       674
Enquadernació:           tapatova
Bestseller
Estil trama:             complexa
Estil vocabulari:        complexa
Quantiat de personatges: pocs
================================================================
Titol:                   Amanecer
Autor:                   Stephenie Meyer
Preu:                    20.0 euros
ISBN:                    607110033X
Gènere:                  terror
Nacionalitat autor:      Estats Units
Any de publicació:       2008
Número de pàgines:       829
Enquadernació:           tapatova
Bestseller
Estil trama:             simple
Estil vocabulari:        simple
Quantiat de personatges: pocs

================================================================

Conclusió
=========

Els llibres recomanats són, casualment, els últims llibres que ha llegit i que li han agradat molt. Diu que sempre s'enganxa a aquest tipus de triologies...



================================================================
################################################################
Xavier  ########################################################
################################################################
================================================================

Perfil
======

El Xavier és professor d'un institut de 57 anys. Sempre ha sigut molt llegidor, tot i que actualment té poc temps i llegeix un parell d'hores setmanals. Li agrada molt la novel·la històrica, tot i que està obert a nous gèneres. No el criden massa els llibres més populars ni els més venuts, i intenta evitar els llibres gruixuts. Els llibres que llegeix han d'estar en català tot i que la procedència de l'autor i li és indiferent. No li importa que el llibre sigui enrevessat, tot i que li molesta que hi hagi masses personatges perquè diu que "es perd". Omple els llibres de notes i usa llibretes també. Demana sempre que la lletra sigui gran perquè diu que "no hi veu". Per ell el preu és un factor important i diu que no es vol gastar més de 20€ en un llibre.



Sistema basat en el coneixement recomanador de llibres

Nom? Xavier

Edat?  (número) 57

Sexe? (home dona) home

Llengua materna? (catala castella altres) catala

Ocupació? (estudiant academic professional desocupat altres) academic

En quina freqüència acostumes a llegir? (diaria setmanal mensual indiferent) setmanal

Quantes hores acostumes a llegir en aquesta freqüència? (número) 2

Gènere:  (classics fantasia ficcio historica negra romantica scifi social terror thriller indiferent) historica

T'agrada experimentar amb nous gèneres? (molt bastant indiferent poc gens) bastant

On llegeixes habitualment? (casa exterior transport indiferent) casa

Prefereixes els llibres ben enquadernats? (molt bastant indiferent poc gens) indiferent

Prefereixes els llibres actuals? (molt bastant indiferent poc gens) bastant

T'agraden els best-sellers? (molt bastant indiferent poc gens) poc

Estàs disposat a llegir llibres gruixuts? (molt bastant indiferent poc gens) poc

Prefereixes que el vocabulari usat sigui senzill? (molt bastant indiferent poc gens) poc

T'agrada prendre notes al costat dels llibres? (molt bastant indiferent poc gens) bastant

T'agraden els llibres amb molts personatges? (molt bastant indiferent poc gens) poc

Quina és la teva preferència per la nacionalitat de l'autor? (catala espanyol catala-espanyol estranger indiferent)indiferent

En quina llengua prefereixes llegir? (catala castella altres indiferent) catala

Autor preferit: 
  1. Cap dels anteriors
  2. Stephenie Meyer
  3. Raymond Carver
  4. Paulo Coelho
  5. Gabriel García Márquez
  6. Thomas Pynchon
  7. Boris Pahor
  8. Julian Barnes
  9. Joseph Roth
  10. Jose Antonio Cebrián
  11. Parramon
  12. Juan Bonilla
  13. Quim Monzó
  14. Josep Pla
  15. Bernhard Schlink
  16. Khaled Hosseini
  17. Glen Cooper
  18. Mark Twain
  19. Roberto Bolaño
  20. Stieg Larsson
  21. Francisco Jimenez
  22. Xavier Montaño
  23. Robert Luis Stevenson
  24. Richard Yates
  25. Maria Àngels Anglada
  26. Xavier Bosch
  27. Camilla Lackberg
  28. Jane Bowles
  29. Thomas Pynchon
  30. Stefano Bordiglioni
  31. Donata Pizzato
  32. Antoine de Saint-Exupéry
  33. Laura Esquivel
  34. Sandra Cisneros
  35. Hector Abad Faciolince
  36. Italo Calvino
  37. Isaac Asimov
  38. Frank Herbert
  39. George Orwell
  40. J. R. R. Tolkien
  41. Clive Staples Lewis
  42. R. A. Salvatore
  43. Jules Verne
  44. Douglas Adams
  45. Orson Scott Card
  46. Arthur C. Clarke
  47. Vizconde de Lascano Tegui
  48. Carme Lafay
  49. J. K. Rowling
  50. Isabel Allende
  51. Dan Brown
  52. Carlos Ruiz Zafón
  53. Alice Sebold
Indica el numero corresponent: 39

Prefereixes que la lletra sigui gran? (molt bastant indiferent poc gens) bastant

El preu és un factor prioritari? (si no indiferent) si

Fins quan estàs disposat a gastar-te en un llibre? (número) 20

Prefereixes els llibres amb contingut explícit (violència o sexe)? (molt bastant indiferent poc gens) indiferent
================================================================

Xavier, 
aquests llibres són la recomanació del SBC: 
================================================================
Titol:                   1984
Autor:                   George Orwell
Preu:                    9.0 euros
ISBN:                    9344444932283
Gènere:                  scifi
Nacionalitat autor:      Regne Unit
Any de publicació:       1949
Número de pàgines:       180
Enquadernació:           tapatova
Contingut explícit
Estil trama:             complexa
Estil vocabulari:        simple
Quantiat de personatges: pocs
================================================================
Titol:                   El caçador d'estels
Autor:                   Khaled Hosseini
Preu:                    8.95 euros
ISBN:                    9788499081106
Gènere:                  historica
Nacionalitat autor:      Persa
Any de publicació:       2006
Número de pàgines:       255
Enquadernació:           butxaca
Bestseller
Estil trama:             complexa
Estil vocabulari:        simple
Quantiat de personatges: pocs
================================================================
Titol:                   L'elegància de quan es dorm
Autor:                   Vizconde de Lascano Tegui
Preu:                    19.0 euros
ISBN:                    978-84-935927-5-2
Gènere:                  social
Nacionalitat autor:      Argentina
Any de publicació:       2008
Número de pàgines:       185
Enquadernació:           tapadura
Estil trama:             simple
Estil vocabulari:        complexa
Quantiat de personatges: pocs

================================================================

Conclusió
========
Els primer llibre és un dels seus preferits, així que es considera un encert. El segon li ha recomanat força gent, i creu que pot ser molt interessant. I el tercer no l'ha sentit anomenar, però les característiques les valora bastant interessants.

\chapter{Conclusions}
%!TEX root = document.tex


\section{Secció}

Fórmula $\Theta(P)$.
Negreta \textbf{solIni2})
Cursiva \emph{recorreguts en cercle}
Referència~\ref{fig:{images/grafic_amb_control_final.png}}
\imatgePetita{images/grafic_sense_control_final.png}{No es té en compte la distància entre la parada candidata i el final, les solucions poden caure en recorreguts en cercle.} 


% \section{Intro}
% 
% \subsection{Blablabla}
% asdfasdf
% 
% \subsubsection{Lol}
% 
% Amb textmate es compila femt pometa-R
% 
% Ah, i has de deixar una línia en blanc entre dos paràgrafs si vols que surtin separats, sinó anirà a continuació...
% 
% Negreta: \textbf{text en negreta}
% 
% Cursiva: \textbf{Text en cursiva}

% Variable o similar: \codi{x.y}
% 
% \begin{lstlisting}[label=compi, caption=Comanda de compilació, language=Java]
% codi
% llarg
% \end{lstlisting}
% 


\chapter{Annex: Entrevista a un llibreter}
%!TEX root = document.tex
\label{entrevista}

Notes: Al final li he fet una entrevista de veritat a un amic (no és el meu pare). Ho completaré amb altres preguntes patillades i d'altres al meu pare.


\textbf{Què és el primer que preguntes quan algú et demana que li recomanis un llibre?}

  Quin tipus de llibre llegeixes normalment. Una pregunta molt bona, considero jo, és quins llibres t'has llegit més d'un cop.

\textbf{I si et diuen que no llegeixen gaire? Per on els faries començar?}

  Home, pots preguntar interessos en altres camps (com la televisió, cine, ...) per fer-te una idea.

\textbf{Recomanes coses diferents als homes i a les dones?}

  Depèn del cas, pero en general sí.

\textbf{Quins d'aquests punts creus que són més importants: l'edat del lector, el sexe, els gèneres que diu que són els seus preferits, l'estona que dedica a llegir, on acostuma a llegir (al llit, al metro, als bars), ...}

  Els gèneres preferits i el temps disponible (bàsicament perquè si hi dedica molta es fàcil que sigui millor lector i li pots recomanar coses mes difícils)

\textbf{Quines altres preguntes creus que ajuden a encertar-la recomanant llibres?}

Preguntaria si com a lector t'agrada que t'enganyin i et portin per terrenys desconeguts, o t'agraden les coses properes i conegudes.

\textbf{Quan algú et diu que és molt d'un gènere en concret, acostumes a intentar allunyar-lo d'aquest i suggerir-li coses diferents? O li fas cas i et cenyeixes al que ell et diu? (relacionat amb la pregunta anterior)}

 Jo li obriria el ventall. Sobretot si és un lector habitual i confio que acceptarà les alternatives.

 Una altra pregunta que crec que diferencia bastant als lectors és si prèn notes en el llibre o no.

\textbf{Creus que el tipus d'enquadernació afecta a l'elecció del llibre?}

  Jo crec que els lectors més reincidents poden ser una mica fetitxistes i preferir llibres de tapa dura, sobretot amb els seus autors preferits o llibres de capçalera. Una mica com passa amb els vinils.

\textbf{I com seria el teu llibre perfecte?}

  Llibre de poesia, tapa dura, un llibre gran però de poemes curts, complex, en català o castellà. la nacionalitat de l'autor no és massa important, però en cas de desconeguts tiro més per un país exòtic que proper. I si pot ser escrit al segle vint.



\end{document}


% Fórmula $\Theta(P)$.
% Negreta \textbf{solIni2})
% Cursiva \emph{recorreguts en cercle}

% Variable o similar: \codi{x.y}
% 
% \begin{lstlisting}[label=compi, caption=Comanda de compilació, language=Java]
% codi
% llarg
% \end{lstlisting}
% 

